\documentclass{article}
\usepackage[T1]{fontenc}
\usepackage[utf8]{inputenc}
\usepackage[english]{babel}

\usepackage{amsmath}
\usepackage{siunitx}
\usepackage{listings}
\usepackage{color}
\usepackage{textcomp}
\usepackage{graphicx}
\usepackage{xr}
\usepackage[section]{placeins}
%\usepackage{hyperref}

\externaldocument{hw2_code}

\definecolor{matlabgreen}{RGB}{28,172,0}
\definecolor{matlablilas}{RGB}{170,55,241}

\newcommand{\includecode}[1]{\lstinputlisting[caption={\ttfamily #1.m},label={lst:#1}]{matlab/#1.m}}
\newcommand{\inlinecode}[1]{\lstinline[basicstyle=\ttfamily,keywordstyle={},stringstyle={},commentstyle={\itshape}]{#1}}

\renewcommand{\vec}[1]{\underline{#1}}
\newcommand{\E}[1]{\operatorname{E}\left[ #1 \right]}
\newcommand{\norm}[1]{\left\lVert#1\right\rVert}
\newcommand{\abs}[1]{\left|#1\right|}

\author{Enrico Polo \and Riccardo Zanol}
\title{Homework 2}

\begin{document}
\lstset{
  language=Matlab,
  basicstyle={\ttfamily \footnotesize},
  breaklines=true,
  morekeywords={true,false,warning,xlim,ylim},
  keywordstyle=\color{blue},
  stringstyle=\color{matlablilas},
  commentstyle={\color{matlabgreen} \itshape},
  numberstyle={\ttfamily \tiny},
  frame=leftline,
  showstringspaces=false,
  numbers=left,
  upquote=true,
}
\maketitle
\section*{Problem 1}
\section*{Problem 2}
\section*{Problem 3}
\end{document}
