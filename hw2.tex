\documentclass{article}
\usepackage[T1]{fontenc}
\usepackage[utf8]{inputenc}
\usepackage[english]{babel}

\usepackage{amsmath}
\usepackage{siunitx}
\usepackage{listings}
\usepackage{color}
\usepackage{textcomp}
\usepackage{graphicx}
\usepackage{xr}
\usepackage[section]{placeins}
%\usepackage{hyperref}

\externaldocument{hw2_code}

\definecolor{matlabgreen}{RGB}{28,172,0}
\definecolor{matlablilas}{RGB}{170,55,241}

\newcommand{\includecode}[1]{\lstinputlisting[caption={\ttfamily #1.m},label={lst:#1}]{matlab/#1.m}}
\newcommand{\inlinecode}[1]{\lstinline[basicstyle=\ttfamily,keywordstyle={},stringstyle={},commentstyle={\itshape}]{#1}}

\renewcommand{\vec}[1]{\underline{#1}}
\renewcommand{\Re}[1]{\operatorname{Re}\left[#1\right]}
\newcommand{\E}[1]{\operatorname{E}\left[#1\right]}
\newcommand{\norm}[1]{\left\lVert#1\right\rVert}
\newcommand{\abs}[1]{\left|#1\right|}

\author{Enrico Polo \and Riccardo Zanol}
\title{Homework 2}

\begin{document}
\lstset{
  language=Matlab,
  basicstyle={\ttfamily \footnotesize},
  breaklines=true,
  morekeywords={true,false,warning,xlim,ylim},
  keywordstyle=\color{blue},
  stringstyle=\color{matlablilas},
  commentstyle={\color{matlabgreen} \itshape},
  numberstyle={\ttfamily \tiny},
  frame=leftline,
  showstringspaces=false,
  numbers=left,
  upquote=true,
}
\maketitle
\section*{Problem 1}
\section*{Problem 2}
To prove equation (3.272) we start from equation (3.70) and assume that:
\begin{itemize}
\item $\vec{\Delta c}(k)$ and $\vec{x}(k)$ are independent, 
\item $e_{min}(k)$ and $\vec{x}(k)$ are orthogonal, 
\item $\vec{x}^T(k)\vec{x}^*(k) \approx Nr_x(0)$.
\end{itemize}
In the LMS algorithm the error on the coefficient vector changes at each iteration in this way (3.70):
\begin{equation*}
  \vec{\Delta c}(k+1) = \left( I - \mu\vec{x}^*(k)\vec{x}^T(k) \right) \vec{\Delta c}(k) + \mu e_{min}(k)\vec{x}^*(k) , 
\end{equation*}
and, if we take the expectation of the squared norm of both terms, we get
\begin{align}
  \E{\norm{\vec{\Delta c}(k+1)}^2} % &=  \E{\norm{\left( I - \mu\vec{x}^*(k)\vec{x}^T(k) \right) \vec{\Delta c}(k) + \mu e_{min}(k)\vec{x}^*(k)}^2} \\
  &= \E{\norm{\left( I - \mu\vec{x}^*(k)\vec{x}^T(k) \right) \vec{\Delta c}(k)}^2}
  \label{eq:exp_1} \\
  & \quad + \E{\norm{\mu e_{min}(k)\vec{x}^*(k)}^2}
  \label{eq:exp_2} \\
  & \quad + \E{\left(\left( I - \mu\vec{x}^*(k)\vec{x}^T(k) \right)\vec{\Delta c}(k)\right)^T\left(\mu e_{min}(k)\vec{x}^*(k)\right)^*}
  \label{eq:exp_3} \\
  & \quad + \E{\left(\mu e_{min}(k)\vec{x}^*(k)\right)^T\left(\left( I - \mu\vec{x}^*(k) \vec{x}^T(k)\right) \vec{\Delta c}(k)\right)^*}
  \label{eq:exp_4} .
\end{align}

If we write the norm term by term the first expectation
(\ref{eq:exp_1}) becomes
\[ \E{\sum_{i=0}^{N-1}\abs{1 - \mu\abs{x(k-i)}^2}^2\abs{\Delta c_i(k)}^2} \]
then we compute the product between the two squared absolute values,
we bring the expectaion inside the finite sum and we exploit the
independece between $x(k-i)$ and $\Delta c_i(k)$ to factor the
expectations so the first term becomes:
\[ \E{\norm{\vec{\Delta c}(k)}^2} +
\mu^2\sum_{i=0}^{N-1}\E{\abs{x(k-i)}^4}\E{\abs{\Delta c_i(k)}^2} - 2\mu
r_x(0)\E{\norm{\vec{\Delta c}(k)}^2}
\]
where we also exploited the WSS of $x(k)$ to write $\E{\abs{x(k-i)}^2}
= r_x(0)$. {\color{red} e adesso? supponiamo l'input bianco? supponiamo qualche forma per il 4o momento?}

If we use the approximation $\norm{\vec{x}(k)}^2 \approx Nr_x(0)$ in
the second expectation (\ref{eq:exp_2}), it becomes
\[
\mu^2 \E{\abs{e_{min}(k)}^2 \norm{\vec{x}^*(k)}^2}
= \mu^2Nr_x(0)\E{\abs{e_{min}(k)}^2}
= \mu^2Nr_x(0)J_{min} .
\]

The last two expectations (\ref{eq:exp_3}) and (\ref{eq:exp_4}) can be
rewritten as, respectively,
\begin{align*}
  & \E{\left(\mu - \mu^2\norm{\vec{x}(k)}^2\right) e_{min}^*(k) \vec{\Delta c}^T(k)\vec{x}(k)} \\
  & \E{\left(\mu - \mu^2\norm{\vec{x}(k)}^2\right) e_{min}(k) \vec{\Delta c}^T(k)\vec{x}(k)}
\end{align*}
{\color{red} anche qua non saprei più come fare}
\section*{Problem 3}
\end{document}
