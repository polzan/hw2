\documentclass[a4paper,oneside]{article}
\usepackage[T1]{fontenc}
\usepackage[utf8]{inputenc}
\usepackage[english]{babel}

\usepackage[margin=2.54cm]{geometry}
\usepackage{amsmath}
\usepackage{siunitx}
\usepackage{listings}
\usepackage{color}
\usepackage{textcomp}
\usepackage{graphicx}
\usepackage{xr}
\usepackage{subcaption}
%\usepackage{changepage}
\usepackage[section]{placeins}
%\usepackage{hyperref}

%\strictpagecheck
\externaldocument{hw2_code}

\definecolor{matlabgreen}{RGB}{28,172,0}
\definecolor{matlablilas}{RGB}{170,55,241}

\newcommand{\includecode}[1]{\lstinputlisting[caption={\ttfamily #1.m},label={lst:#1}]{matlab/#1.m}}
\newcommand{\inlinecode}[1]{\lstinline[basicstyle=\ttfamily,keywordstyle={},stringstyle={},commentstyle={\itshape}]{#1}}

\renewcommand{\vec}[1]{\underline{#1}}
\renewcommand{\Re}[1]{\operatorname{Re}\left[#1\right]}
\newcommand{\E}[1]{\operatorname{E}\left[#1\right]}
\newcommand{\norm}[1]{\left\lVert#1\right\rVert}
\newcommand{\abs}[1]{\left|#1\right|}
\newcommand{\F}[1]{\operatorname{\mathcal{F}}\left[#1\right]}
\newcommand{\ceil}[1]{\left\lceil#1\right\rceil}

\author{Enrico Polo \and Riccardo Zanol}
\title{Homework 2}

\begin{document}
\lstset{
  language=Matlab,
  basicstyle={\ttfamily \footnotesize},
  breaklines=true,
  morekeywords={true,false,warning,xlim,ylim},
  keywordstyle=\color{blue},
  stringstyle=\color{matlablilas},
  commentstyle={\color{matlabgreen} \itshape},
  numberstyle={\ttfamily \tiny},
  frame=leftline,
  showstringspaces=false,
  numbers=left,
  upquote=true,
}
\maketitle
\section*{Problem 1}
To estimate the filter impulse response $\vec{h}^T = [h(0), ..., h(N-1)]$ by using a FIR filter of order $N_{fil}$ we use a PN sequence $\vec{p}^T = [p(0),...p(L-1)]$ (taking values in the set \{-1;1\}) periodic of period $L\geq N$ with $N=15$ that is the upper bound of the order of $h$. To take into account the transient of the filter in analysis we partially repeat the input sequence adding N-1 samples, putting as input a sequence $\vec{x}^T = [p(0), p(1),...,p(L-1),p(0),...,p(N-2))]$. So we get that the output of the unknown system (including the noise) is given by 
\begin{align*}
d(k)& = \vec{h}^T \vec{x} + w(k)&  k& = N-1,N-2...,N-1+L-1
\end{align*}
At the receiver side we consider $d(k)$ as the desired process we want to trace using our FIR filter, using $\vec{p}$ as input. In this way we can find that setting properly $N_{fil}$ and $L$ that $\vec{c}_{opt} = \vec{h}$ in the ideal case (same realisation of the white input at the sender and at the receiver) and approximates it using a PN sequence. 
\newline Applying the correlation method we can find an estimate of the filter coefficients by using the relation 
\begin{align*}
\hat{r_{dx}}(n) &= \frac{1}{L} \sum_{k=N-1}^{(N-1) + (L-1)} {d(k)p^*(k-n)} & n&=0,1,...N_{fil} -1
\end{align*}
in which $d(k)$ is a received sequence of length L and $p(k)$ is the input PN sequence of length L, known by the receiver. If we try an $N_{fil} > N$ the index of $p(k-n)$ must be negative in the computation of the sum but we can already compute that by using the property that for periodicity $p(-i) = p(L-i)$. 

Using instead the LS method we compute the autocorrelation matrix of the PN sequence $\phi$ whose elements are 
\begin{align*}
\phi(i,n) &= \sum_{k=N-1}^{(N-1) + (L-1)}{p(k-i)p^*(k-n)}&
i,n&=0,1,...N_{fil} -1
\end{align*}
and the cross correlation vector between $\vec{d}$ and $\vec{p}$, $\vec{\theta}^T = [\theta(0), \theta(1),..., \theta(N_{fil}-1)]$ whose elements are
\begin{align*}
\theta(n)& = \sum_{k=N-1}^{(N-1) + (L-1)}{d(k)p^*(k-n)}& n=0,1...,N_{fil}-1
\end{align*}
Then we find te optimal solution for the LS problem that give us  $\vec{\hat{h_{ls}}} = \vec{\phi}^{-1} \vec{\theta}$ with the correspondent error 
$ e_{min} = e_d - \vec{\theta}^H \vec{\hat{h}}_{ls}$ ($e_d$ energy of the desired signal $d(k)$). This relation is valid also for the correletion method error by using $\vec{\hat{h}}_{corr}$ instead of $\vec{\hat{h}}_{ls}$ and computing also in that case the vector $\vec{\theta}$.  Using this, we can finally estimate the power of the noise that affects $d(k)$ by $\hat{\sigma_\omega^2} = \frac{e_{min}}{L}$.
\newline In our tries we tried to find the optimal values $N_{fil}$ and $L$. We started from $L = 15$ increasing it until the error $e$ didn't increase any longer. With both methods we find that from L=63 or L=127, it depends on the specific realization of w(k), and $N_{fil}$ fixed, we get the same results in therms of $\hat{\sigma_\omega^2} = e/L\simeq -5 dB$. Regarding the estimation of the impulse response we have to increase the value of $L$ at least to 255, to have quite good results as can be seen in Fig. ~\ref{plot:h}.
\begin{figure}[htbp]
  \centering
  \includegraphics[width=0.7\textwidth]{matlab/h_r8}
  \caption{Impulse response: theoretical and estimated}
    \label{plot:h}
\end{figure}
So finally we decided to stop here in order to find a trade-off between the computation time and good accuracy. Furthermore we observe that, as espected, the variance on the estimation of $h$ and $\hat{\sigma_\omega^2}$ decreases while $L$ becomes larger and also for this reason we decided for $L=255$ samples.
We analysed, as required, also the behaviour of $e/L$ respect to $N_fil$ for both methods (Fig. ~\ref{plot:eL}). We see that the $\hat{\sigma_\omega^2 = \frac{e}{L}}$ is higher than the real value. This is due to the fact that we are using a number $N_fil$ of coefficients that is not enough to correctly estimate the shape of $h$ and we are adding an error to that due only to the presence of $w(k)$. Then we have a good estimation interval in which $N_fil \simeq N$ and if the number of coefficients used for the estimation is too large $N_fil \longrightarrow L$ we are tracing the whole process so the error decreases and one more time we are not estimating correctly $\sigma_\omega^2$.
\newline This behaviour can be seen also observing the beahviour of $\Lambda_n$. In Fig. ~\ref{plot:snrCORR} is reported the estimate SNR using the correlation method $\Lambda_n = \frac{\sigma_\omega^2}{M_p || h-\hat{h}_{corr} ||^2}$, supposing we know $h$ and $\sigma_\omega^2$ and having that for a PN sequence $M_p = 1$. The red lines represent the ideal behaviour compute as $\Lambda_n^{corr} = \frac{L}{N_{fil} + 1/L \bigl[ \Lambda + (N_{fil}-2)\bigl(\frac{|H(0)|^2}{\sigma_\omega^2} \bigr)\bigr]}$ in which $\Lambda = \frac{M_p||\vec{h}||^2}{\sigma_\omega^2}$  and and $|H(0)|^2 = \sum_{i=0}^{N-1} h_i$. 
  The same was done for the LS method ~\ref{plot:snrLS}) in which we have used $\Lambda_n = \frac{\sigma_\omega^2}{M_p || h-\hat{h}_ls ||^2}$ and $\Lambda_n^{ls} = \frac{(L+1)(L+1-N_{fil})}{N_{fil}(L+2-N_{fil})}$.
  In fact in both figures we see that for $N_fil$ too low the $\hat{\Lambda_n}$ is significantly smaller than the theoretical one (that assume we have enough coefficients to correctly model $h$). Then it increases and has a peak in $N_{fil} \simeq N$ before starting decreasing following the ideal shape.  So in conclusion if we want to find an optimal $N_{fil}$ we must select a value near $N$. By analizing the SNR plots the peak is on $N_{fil} = 12$ so we can select this value. This is true in our case because the $h$ we are estimating from n = 13 to n = 15 is very close to zero so in fact we commit an higher error by trying to estimate the last 3 coefficients than to set them to zero (as can be seen in Fig. 1). We could also increase $L$ to a very high value to better estimate the shape of $h$ also in the tail (and we see that the peak moves to $N_{fil} = N$) but the computation time will become too high. But if we are trying to estimate an unknown system with a different behaviour we should select $N_{fil} = N$ in order to have a coefficient for each $h_i$. 
  The estimeted parameters are: $\hat{\sigma_\omega^2} = 0.2658$ with LS and $\hat{\sigma_\omega^2} = 0.27$ with the correlation method. We report also the $\hat{h}_i$ in the following tab for $L = 255$ and $N_{fil} = 15$.
  
\begin{table}
\centering
  	 												 
\begin{tabular}{|c|c|c|}
\hline
i & $ \hat{h}_i^{corr} $            & $\hat{h}_i^{ls} $          \\
\hline
0 &	0,9304	& 0.9373\\
\hline
1 &	0,6516	& 0.6596\\
\hline
2 &	0,4757	& 0.4844\\
\hline
3 &	0,2877	& 0.2972\\
\hline
4 &	0,1031	& 0.1133\\
\hline
5 &	0,0500	& 0.0604\\
\hline
6 &	0,0859	& 0.0961\\
\hline
7 &	0,0505	& 0.0608\\
\hline
8 &	-0,0178	& -0.0073\\
\hline
9 &	0,0496	& 0.0600\\
\hline
10 &	 -0,0579	& -0.0472\\
\hline
11 &	 0,0287	& 0.0391\\
\hline
12 &	 -0,0266 	& -0.0161\\
\hline
13 &	 -0,0278 	&-0.0172\\
\hline
14 &	 -0,0407 	& -0.0300\\
\hline
\end{tabular}
\end{table}
\begin{figure}[htbp]
  \centering
  \includegraphics[width=0.7\textwidth]{matlab/eL}
  \caption{$\frac{e}{L}$ (dB) as function of $N_{fil}$}
    \label{plot:eL}
\end{figure}

\begin{figure}[htbp]
  \centering
  \includegraphics[width=0.7\textwidth]{matlab/eCORR}
  \caption{$\Lambda_n^{corr}$ (dB) as function of $N_{fil}$}
    \label{plot:snrCORR}
\end{figure}
\begin{figure}[htbp]
  \centering
  \includegraphics[width=0.7\textwidth]{matlab/eLS}
  \caption{$\Lambda_n^{ls}$ (dB) as function of $N_{fil}$}
  
    \label{plot:snrLS}
\end{figure}
%$\Lambda_n = \frac{\sigma_\omega^2}{M_x || h-\hat{h} ||^2}11111$
%$\Lambda_n^{corr} = \frac{L}{N_{fil} + 1/L \bigl[ \Lambda + (N_{fil}-2)\bigl(\frac{|H(0)|%^2}{\sigma_\omega^2} \bigr)\bigr]}$
%$\Lambda_n^{ls} = \frac{(L+1)(L+1-N_{fil})}{N_{fil}(L+2-N_{fil})}$
%$\Lambda = \frac{M_x||\vec{h}||^2}{\sigma_\omega^2}111$


\section*{Problem 2}
To prove equation (3.271) of the textbook we start from the definition
of $J(k) = \E{\abs{e(k)}^2}$ and substitute
\[ e(k) = d(k) - y(k) = \left(\vec{h}^T - \vec{c}^T(k) \right) \vec{x}(k) + w(k) \]
inside the expectation. So the mean squared error becomes
\begin{equation}
  J(k) = \E{\abs{\left(\vec{h}^T - \vec{c}^T(k) \right) \vec{x}(k)}^2}
  + \E{\abs{w(k)}^2}
  + \E{2\Re{\left(\vec{h}^T - \vec{c}^T(k) \right) \vec{x}(k) w^*(k)}}
  \label{eq:J}
\end{equation}
where the last term is zero because the random processes inside it are
assumed to be independent and the mean of $w(k)$ is zero.

The second term of equation (\ref{eq:J}) is just $\sigma^2_w$, while
the first expectation can be written as
\[ \E{
  \left(\sum_{i=0}^{N_{max}}\left(h_i - c_i\right)x(k-i)\right)
  \left(\sum_{j=0}^{N_{max}}\left(h_j - c_j\right)x(k-j)\right)^* } \]
where $N_{max} = \max\{N, N_h\}$ and we assume that all the vectors
are padded with zeros to length $N_{max}$.  If we rearrange the sums
and the expectation then, if we assume that the pairs $(c_i,c_j)$ and
$(x(k-i),x(k-j))$ are statistically independent $\forall i,j$, we can
factor the expectation:
\begin{equation}
%%   &\sum_{i=0}^{N_{max}}\sum_{j=0}^{N_{max}}
%% \E{(h_i -c_i)(h_j-c_j)^*x(k-i)x^*(k-j)} \\
\sum_{i=0}^{N_{max}}\sum_{j=0}^{N_{max}}
\E{(h_i -c_i)(h_j-c_j)^*}
\E{x(k-i)x^*(k-j)} .
\label{eq:J_sums}
\end{equation}
Now if the input $x(k)$ is white its autocorrelation is
\[
\E{x(k-i)x^*(k-j)} = r_x(j-i) = r_x(0)\delta(j-i)
\]
so the only non-zero terms of equation (\ref{eq:J_sums}) are those
with $i=j$ and it becomes:
\begin{equation}
  r_x(0)\E{\norm{\vec{h}-\vec{c}(k)}^2}
  \label{eq:J_final_diff}
\end{equation}

Depending on the order of the channel and the filter there are two cases:
\begin{enumerate}
\item If $N \geq N_h$ the optimal value of $\vec{c}(k)$ is
  $\vec{c}_{opt} = \vec{h}$, so the difference in equation
  (\ref{eq:J_final_diff}) can be written as $\vec{h}-\vec{c}(k) =
  -\left(\vec{c}(k)-\vec{c}_{opt}\right) = -\vec{\Delta c}(k)$ and the
  MSE is:
  \begin{equation}
    J(k) = \sigma^2_w + r_x(0)\E{\norm{\vec{\Delta c}(k)}^2}
    = J_{min} + r_x(0)\E{\norm{\vec{\Delta c}(k)}^2}.
  \end{equation}
  \item If $N < N_h$ the filter can only model the first $N$
    coefficients of the channel and the optimal solution is
    $\vec{c}_{opt} = [h_0,h_1,\dots h_{N-1},0,\dots 0]^T$. If we write
    the difference of equation (\ref{eq:J_final_diff}) using the
    residual error vector $ \vec{\Delta h}(\infty) =
    [0,\dots0,-h_{N},-h_{N+1},\dots-h_{N_h-1}]^T $ we get
    $\vec{h}-\vec{c}(k) = -\vec{\Delta c}(k) -\vec{\Delta h}(\infty)$.
    Computing the expectation of equation (\ref{eq:J_final_diff}) by
    writing the norm term by term and bringing out the sum we get
    \[
    \sum_{i=0}^{N_h}\E{
      \abs{\Delta h_i(\infty)}^2
      + \abs{\Delta c_i(k)}^2
      + 2\Re{\Delta h_i(\infty) \Delta c_i^*(k)}
    }
    \]
    where we notice that, in the last term, one of either $\Delta
    h_i(\infty)$ or $\Delta c_i(k)$ must be zero, so what is left is
    $\norm{\vec{\Delta h}(\infty)}^2 + \E{\norm{\vec{\Delta c}(k)}^2}$
    and, putting everything together, we get the MSE as
    \begin{equation}
      J(k) = \sigma^2_w + r_x(0)\norm{\vec{\Delta h}(\infty)}^2 + r_x(0)\E{\norm{\vec{\Delta c}(k)}^2}
      = J_{min} + r_x(0)\E{\norm{\vec{\Delta c}(k)}^2} .
    \end{equation}
\end{enumerate}

To prove equation (3.272) of the textbook we start from the equation
that gives the change in $\vec{\Delta c}(k)$ at each iteration and
assume that:
\begin{itemize}
\item $\vec{\Delta c}(k)$ and $\vec{x}(k)$ are independent, 
\item $e_{min}(k)$ and $\vec{x}(k)$ are orthogonal, 
\item $\vec{x}^T(k)\vec{x}^*(k) \approx Nr_x(0)$.
\end{itemize}
In the LMS algorithm the error on the coefficient vector changes at each iteration in this way (equation (3.70) of the textbook):
\begin{equation*}
  \vec{\Delta c}(k+1) = \left( I - \mu\vec{x}^*(k)\vec{x}^T(k) \right) \vec{\Delta c}(k) + \mu e_{min}(k)\vec{x}^*(k) , 
\end{equation*}
and, if we take the expectation of the squared norm of both terms, we get
\begin{align}
  \E{\norm{\vec{\Delta c}(k+1)}^2} % &=  \E{\norm{\left( I - \mu\vec{x}^*(k)\vec{x}^T(k) \right) \vec{\Delta c}(k) + \mu e_{min}(k)\vec{x}^*(k)}^2} \\
  &= \E{\norm{\left( I - \mu\vec{x}^*(k)\vec{x}^T(k) \right) \vec{\Delta c}(k)}^2}
  \label{eq:exp_1} \\
  & \quad + \E{\norm{\mu e_{min}(k)\vec{x}^*(k)}^2}
  \label{eq:exp_2} \\
  & \quad + \E{\left(\left( I - \mu\vec{x}^*(k)\vec{x}^T(k) \right)\vec{\Delta c}(k)\right)^T\left(\mu e_{min}(k)\vec{x}^*(k)\right)^*}
  \label{eq:exp_3} \\
  & \quad + \E{\left(\mu e_{min}(k)\vec{x}^*(k)\right)^T\left(\left( I - \mu\vec{x}^*(k) \vec{x}^T(k)\right) \vec{\Delta c}(k)\right)^*}
  \label{eq:exp_4} .
\end{align}

If we write the norm term by term the first expectation
(\ref{eq:exp_1}) becomes
\[ \E{\sum_{i=0}^{N-1}\abs{1 - \mu\abs{x(k-i)}^2}^2\abs{\Delta c_i(k)}^2} \]
then we compute the product between the two squared absolute values,
we bring the expectation inside the finite sum and we exploit the
independence between $x(k-i)$ and $\Delta c_i(k)$ to factor the
expectations so the first term becomes:
\[ \E{\norm{\vec{\Delta c}(k)}^2} +
\mu^2\sum_{i=0}^{N-1}\E{\abs{x(k-i)}^4}\E{\abs{\Delta c_i(k)}^2} - 2\mu
r_x(0)\E{\norm{\vec{\Delta c}(k)}^2}
\]
where we also exploited the WSS of $x(k)$ to write $\E{\abs{x(k-i)}^2}
= r_x(0)$. Now the fourth moment of $x(k)$ would be $3\sigma^4_x$, but
we approximate it with $r_x(0)^2$ like in equation (3.78) of the
textbook because we would otherwise get slightly different results.
So in the end we have the following expression:
\begin{equation}
\E{\norm{\vec{\Delta c}(k)}^2} +
\mu^2r_x(0)^2\E{\norm{\vec{\Delta c}(k)}^2} - 2\mu
r_x(0)\E{\norm{\vec{\Delta c}(k)}^2} .
\label{eq:c_final_1}
\end{equation}

If we use the approximation $\norm{\vec{x}(k)}^2 \approx Nr_x(0)$ in
the second expectation (\ref{eq:exp_2}), it becomes
\begin{equation}
\mu^2 \E{\abs{e_{min}(k)}^2 \norm{\vec{x}^*(k)}^2}
= \mu^2Nr_x(0)\E{\abs{e_{min}(k)}^2}
= \mu^2Nr_x(0)J_{min} .
\label{eq:c_final_2}
\end{equation}

The last two expectations (\ref{eq:exp_3}) and (\ref{eq:exp_4}) can be
rewritten as, respectively,
\begin{align*}
  & \E{\left(\mu - \mu^2\norm{\vec{x}(k)}^2\right) e_{min}^*(k) \vec{\Delta c}^T(k)\vec{x}(k)} \\
  & \E{\left(\mu - \mu^2\norm{\vec{x}(k)}^2\right) e_{min}(k) \vec{\Delta c}^T(k)\vec{x}(k)}
\end{align*}
and they are both zero because of the independence between $\vec{x}(k)$
and $\vec{\Delta c}(k)$ and of the orthogonality of $\vec{x}(k)$ and
$e_{min}(k)$.

Putting together equations (\ref{eq:c_final_1}) and
(\ref{eq:c_final_2}) we get the difference equation
\begin{equation}
  \E{\norm{\vec{\Delta c}(k+1)}^2} = \gamma \E{\norm{\vec{\Delta
        c}(k)}^2} + \mu^2Nr_x(0)J_{min}
\end{equation}
where we define $\gamma$ from equation (\ref{eq:c_final_1}) to be 
the sum of the constant coefficients of $\E{\norm{\vec{\Delta c}(k)}^2}$:
\[ \gamma = 1 + \mu^2r_x(0)^2 - 2\mu r_x(0) . \]
The solution to this difference equation for $k \geq 0$ is
\begin{equation}
  \E{\norm{\vec{\Delta c}(k)}^2} =
  \gamma^k \E{\norm{\vec{\Delta c}(0)}^2} +
  \mu^2Nr_x(0)J_{min} \frac{1 - \gamma^k}{1 - \gamma}
\end{equation}
which is what we wanted to prove.
\section*{Problem 3}
To have the desired exponential shape the power delay profile of the
channel is expressed by
\begin{equation}
  M(\tau) = \frac{1}{\overline{\tau}_{rms}} e^{-\tau / \overline{\tau}_{rms}}
\end{equation}
where the average rms delay spread is $\overline{\tau}_{rms} =
0.2T$. We sample the PDP with period $T_c = T/4$ and divide it by
$\sum_{i=0}^{N_h-1}M(iT_c)$ in order to have
\[ \sum_{i=0}^{N_h-1}\E{\abs{h_i}^2} = 1 . \]
The normalized PDP is shown in Fig.~\ref{plot:pdp} and the values it
takes are reported in Tab.~\ref{tab:pdp}.
\begin{figure}[htbp]
  \centering
  \includegraphics[width=0.5\textwidth]{matlab/plot_pdp}
  \caption{Normalized PDP}
  \label{plot:pdp}
\end{figure}
\begin{table}[h]
  \centering
  \begin{tabular}{>{$}c<{$}>{$}c<{$}}
    M(0T_c) & \SI{-1.458}{\dB} \\
    M(1T_c) & \SI{-6.886}{\dB} \\
    M(2T_c) & \SI{-12.315}{\dB} \\
    M(3T_c) & \SI{-17.744}{\dB} \\
    M(4T_c) & \SI{-23.172}{\dB} \\
  \end{tabular}
  \caption{Values of the normalized PDP}
  \label{tab:pdp}
\end{table}
From the Rice factor we can get the LOS component of $h_0$ as 
\[ C = \sqrt{K/(K+1)} \approx 0.78 \]
because we normalized the PDP.
%, so the statistical power of $h_0$ is
%\[ \E{\abs{h_0}^2} = \E{\abs{\tilde{h}_0}^2} + C^2 \] = 

The classical Doppler spectrum is given by the expression
\begin{equation}
  D(\lambda) = \begin{cases}
    \frac{1}{\pi f_d} \frac{1}{\sqrt{1 - (\lambda/f_d)^2}} \qquad & \text{if} \quad \abs{\lambda} \leq f_d \\
    0 & \text{otherwise}
  \end{cases}
  \label{eq:D}
\end{equation}
where, in our case, $f_d = \frac{25\cdot10^{-5}}{T_c}$.

From equation (4.236) of the textbook
\[ r_g(\Delta t;\tau) = \mathrm{d}(\Delta t)M(\tau) \]
we can see that if we generate a random process with PSD $D(\lambda) =
\F{\mathrm{d}(\Delta t)}$ and multiply it by $M(\tau)$, we have a
random process with the same power spectral density of the channel
impulse response $g(t;\tau)$. So we simulate the taps of the channel
model of section 4.4.6 of the textbook using the schema which that
section suggests:
\begin{itemize}
  \item We generate $N_h$ complex Gaussian signals
    $\overline{w}_i(lT_p)$ with zero mean, power
    $\sigma^2_{\overline{w}_i} = 1$, time quantum $T_p = MT_c$ (see
    (\ref{eq:M})) and length $K_p$ (see (\ref{eq:Kp})).
  \item Each $\overline{w}_i(lT_p)$ is filtered by a filter whose
    frequency response $ H_{ds}(\lambda) \approx \sqrt{D(\lambda)} $
    sets the PSD of the output signal $g'_i(lT_p)$ to
    \[ \mathcal{P}_{g'_i}(\lambda) = \sigma^2_{\overline{w}_i} \abs{H_{ds}(\lambda)}^2 \approx D(\lambda) . \]
    The coefficients of the IIR filter $h_{ds}$ are given in Table 4.6
    of the textbook for the case $f_dT_p = 0.1$ so we must choose the
    interpolation factor
    \begin{equation}
      \frac{T_p}{T_c} = M = \frac{0.1}{f_dT_c}
      \label{eq:M}
    \end{equation}
    in order to have the Doppler spread we want. The output of the
    filter $g'_i(lT_p)$ must be normalized to have statistical power 1,
    because we will set it later using the PDP, so we scale it by $
    \frac{1}{\sqrt{E_{H_{ds}}}} =
    \frac{1}{\sqrt{\frac{1}{K_H}\sum_{n=0}^{K_H}
        \abs{H_{ds}(n/T_c)}^2}} $. We must also discard the first
    samples because they depend on the initial conditions of the
    filter, since the filter is IIR it does not have a finite length
    but we can say that after $L_{ds} = 250$~samples the impulse
    response is zero, since its absolute value is lower than
    $8\cdot10^{-6}$.
    \item Now we must interpolate the generated signals to get to the
      desired time quantum $T_c$, we do so by using a built-in Matlab
      function that increases the sampling frequency by a factor of
      $M$ and then applies a low-pass, linear-phase, FIR filter of
      length $L_{int} = 8M+1$ samples. The filter also has a pass-band
      gain of $M$ so it does not modify the signal's statistical
      power. Since we are applying another filter we must remove the
      transient response of length $L_{int}$ samples from the
      interpolated signal.

      The total number of samples $K_p$ that must be generated in
      order to simulate the channel response for times $kT_c :
      k=0,1,\dots K-1$, taking into account the transients that must
      be dropped, is
      \begin{equation}
        K_p = \ceil{K/M} + L_{ds} + \ceil{L_{int}/M} .
        \label{eq:Kp}
      \end{equation}
      We set the length of the channel response to be simulated to
      $K=80000$ to use the same data for the plots (where we truncate
      it to 12000 samples) and the pdf estimation.
      \item The output of the interpolator $g'_i(kT_c) : i=1,2,\dots
        N_h-1$ is then scaled according to the values of the PDP to
        obtain the channel response coefficients (except for the
        first, $h_0$):
        \begin{equation}
          h_i(kT_c) = g'_i(kT_c) \sqrt{M(iTc)} \qquad i=1,2,\dots N_h-1 .
        \end{equation}
        The first coefficient must also include the constant LOS gain
        $C$ but, to avoid modifying the average transmitted power, we
        must take it into account when we set the power
        $\E{\abs{h_0}^2}$. So the simulated coefficients of the
        channel response are:
        \begin{equation}
          h_i(kT_c) = \begin{cases}
            g'_i(kT_c) \sqrt{M(iTc)} \qquad & i=1,2,\dots N_h-1 \\
            g'_0(kT_c) \sqrt{M(0) - C^2} + C \qquad & i=0
            \end{cases}
        \end{equation}
        which are plotted in Fig.~\ref{plot:h_coeff} for $i=0,2,4$.
\end{itemize}
Since the channel is slow fading the shape of the plots is smooth
because the channel response changes slowly in time, as can be seen in
the figure.
%% \checkoddpage
  %% \ifoddpage
  %%   \newcommand{\figangle}[0]{angle\=90}
  %%   \else
  %%   \newcommand{\figangle}[0]{angle\=270}
  %%   \fi
\begin{figure}[p]
  \centering
  \includegraphics[angle=90,height=\textheight]{matlab/plot_h_coeff}
  \caption{Simulated channel impulse response}
  \label{plot:h_coeff}
\end{figure}

We estimate the probability density function of
$\abs{h_0}/M_{\abs{h_0}}$ by computing an histogram with 2048 bars,
normalizing it and scaling it by the step size. The theoretical pdf
would be a Rice distribution with parameter $K$ if the Doppler spread was $f_d =
0$. We don't know how to compute it in the case $f_d > 0$ so we
compare the Rice pdf with the estimated pdf in Fig.~\ref{plot:pdf},
where we can see that the two curves look similar and the effect of
the Doppler spread are the peaks that we see in the estimated pdf.
\begin{figure}[htbp]
  \centering
  \includegraphics[width=0.7\textwidth]{matlab/plot_pdf}
  \caption{Probability density function of $\abs{h_0}/M_{\abs{h_0}}$}
  \label{plot:pdf}
\end{figure}

If we try to estimate the PSD of $h_0$ with the Welch method we see
that it only detects the spectral line in 0 due to the LOS gain. This
is because the Doppler spectrum is very narrow for the value of
$f_dT_c$ of our case and, if we try to increase it, we see that the
estimated PSD starts to have a flat region and a peak in
correspondence with those of the theoretical PSD.  The theoretical PSD
is $D(\lambda)$ from equation (\ref{eq:D}) plus the spectral line due
to the LOS gain:
\begin{equation}
  \mathcal{P}(\lambda) = D(\lambda) + C^2\delta(\lambda)
\end{equation}
In Fig.~\ref{plot:psd} the theoretical PSD is compared to the one
estimated with the Welch method using a rectangular window, chunks of
length $D=5000$ and overlap $S=1200$. In Fig.~\ref{plot:psd_detail} we
can see a detail of what happens near $f_d$.
\begin{figure}[htbp]
  \centering
  \begin{subfigure}{0.7\textwidth}
    \centering
    \includegraphics[width=\textwidth]{matlab/plot_psd}
  \caption{All the frequencies}
  \label{plot:psd}
  \end{subfigure}
  \begin{subfigure}{0.7\textwidth}
    \centering
    \includegraphics[width=\textwidth]{matlab/plot_psd_detail}
  \caption{Detail near $f_d$}
  \label{plot:psd_detail}
  \end{subfigure}
  \caption{Power spectral density of $h_0$}
\end{figure}
\end{document}
